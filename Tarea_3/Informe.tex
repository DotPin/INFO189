\documentclass{article}
\usepackage[utf8]{inputenc}
\usepackage[english]{babel}
\selectlanguage{english}
\usepackage{amsmath}

\title{Proyecto 1. Formulación Galerkin: Matríces de elementos: }
\author{Sebastian Matamala, Nicolas Jimenez, Diego Rojas, Arturo Reyes}
\date{9 de julio de 2018}

\usepackage{natbib}
\usepackage{graphicx}
\usepackage{booktabs}

\begin{document}

\maketitle


\newpage


%\textbf{Nombre:}
\section{Resolución}

Al correr el programa se obtiene el siguiente sistema de ecuaciones lineales

\begin{figure}[!ht]{sistema}
    \centering
    \[
    \begin{bmatrix}
        {0.014} & {-0.014} & {0.0} & {0.0}\\
        {-0.014} & {0.028} & {-0.014} & {0.0}\\
        {0.0} & {-0.014} & {0.028} & {-0.014}\\
        {0.0} & {0.0} & {-0.014} & {0.014}\\
    \end{bmatrix}
    *
    \begin{bmatrix}
        {\Phi_0}\\
        {\Phi_1}\\
        {\Phi_2}\\
        {\Phi_3}
    \end{bmatrix}
    =
    \begin{bmatrix}
        {0}\\
        {0}\\
        {0}\\
        {0}
    \end{bmatrix}
    \]
    \caption{Matriz de sistema lineal}

\end{figure}
Y resolviendo el sistema de ecuaciones lineales usando los valores de D, L y frontera, usando los valores por defecto se obtiene:
\begin{figure}[!h]
    
    \[
        \begin{bmatrix}{lineales}
            {\Phi_0 = 20}\\
            {\Phi_1 = 20}\\
            {\Phi_2 = 20}\\
            {\Phi_3 = -15}
        \end{bmatrix}
    \]
    \caption{Vector de Solucion}
\end{figure}

Desarrollo de las matrices por método directo.

\section{Programación}

\subsection{Descripción General}
Este informe describira el programa realizado para solucionar el problema descrito en el enunciado.\\
El programa consta de varias funciones que en conjunto sirven para solucionar un problema de elementos finitos en una dimension.

\subsection{Función Elemento}
La siguiente figura describe la representacion matricial de un elemento de la figura.
\begin{figure}[!h]
    \begin{equation}
        R^{(e)}=\left[ \begin{array}{cc}
            \Phi i  \\
            \Phi j
            \end{array}\right]+\frac{D}{L}
            \left[\begin{array}{cccc}
                1 & 1 \\
                1 & 1
                \end{array}
            \right]
            \left\lbrace \begin{array}{cc}
                \Phi i  \\
                \Phi j
                \end{array}\right\rbrace
                *
                \frac {QL}{2} \left[ \begin{array}{cccc}
                    1  \\
                    1
                    \end{array}\right]
        \end{equation}
        \caption{Funcion de elemento}
    \end{figure}

\subsection{Función Global}
watafak
\begin{figure}[!h]
    \[
        \begin{bmatrix}{global}
            {0.014*\Phi_0 - 0.014*\Phi_1}\\
            {-0.014*\Phi_0 + 0.028*\Phi_1-0.014\Phi_2}\\
            {-0.014*\Phi_1 + 0.028*\Phi_2 - 0.014*\Phi_3}\\
            {-0.014*\Phi_2 + 0.014*\phi_3}
        \end{bmatrix}
    \]
    \caption{Matriz global de ecuaciones lineales}
    \end{figure}

\subsection{Función Valores Nodales}
megaderp
\begin{table}[!h]
    \centering
        \begin{tabular}{@{}lllll@{}}
        \toprule
        $e$ & $i$ & $j$ & D/L    & QL/2 \\ \midrule
        1 & 1 & 2 & 1/65   & 0    \\
        2 & 2 & 3 & 1/1600 & 0    \\
        3 & 3 & 4 & 7/500  & 0    \\ \bottomrule
    \end{tabular}
    \caption{Tabla de valores nodales}
\end{table}
\subsection{Función Conductividad}
ultrarage
\begin{figure}[!h]
    \begin{equation}
        N_{e}=
        \left(
            \frac {D}{L}\right) ^{(e)}
        \left[ \begin{array}{cccc}
            1 & 0 & 0 & 0 \\
            0 & 1 & 0 & 0 \\
            0 & 0 & 1 & 0 \\
            0 & 0 & 0 & 1
            \end{array}\right]
    \end{equation}
\end{figure}

\end{document}