\documentclass{article}
\usepackage[utf8]{inputenc}
\usepackage[english]{babel}
\selectlanguage{english}
\usepackage{amsmath}

\title{Proyecto 1. Formulación Galerkin: Matríces de elementos: }
\author{Sebastian Matamala, Nicolas Jimenez, Diego Rojas, Arturo Reyes}
\date{9 de julio de 2018}

\usepackage{natbib}
\usepackage{graphicx}

\begin{document}

\maketitle

%\textbf{Nombre:}
\section{Resolución}

\begin{figure}[!ht]
    \centering
    \[
    \begin{bmatrix}
        {0.014} & {-0.014} & {0.0} & {0.0}\\
        {-0.014} & {0.028} & {-0.014} & {0.0}\\
        {0.0} & {-0.014} & {0.028} & {-0.014}\\
        {0.0} & {0.0} & {-0.014} & {0.014}\\
    \end{bmatrix}
    \]
    \caption{Matriz de Resolucion}
\end{figure}
Desarrollo de las matrices por método directo.

\section{Programación}

\subsection{Descripción General}
Este informe describira el programa realizado para solucionar el problema descrito en el enunciado
\subsection{Función Elemento}
\subsection{Función Global}
\subsection{Función Valores Nodales}
\subsection{Función Conductividad}

\end{document}